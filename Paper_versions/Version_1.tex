%%%%%%%% ICML 2022 EXAMPLE LATEX SUBMISSION FILE %%%%%%%%%%%%%%%%%

\documentclass[nohyperref]{article}

% Recommended, but optional, packages for figures and better typesetting:
\usepackage{microtype}
\usepackage{graphicx}
\usepackage{subfigure}
\usepackage{booktabs} % for professional tables

% hyperref makes hyperlinks in the resulting PDF.
% If your build breaks (sometimes temporarily if a hyperlink spans a page)
% please comment out the following usepackage line and replace
% \usepackage{icml2022} with \usepackage[nohyperref]{icml2022} above.
\usepackage{hyperref}


% Attempt to make hyperref and algorithmic work together better:
\newcommand{\theHalgorithm}{\arabic{algorithm}}

% Use the following line for the initial blind version submitted for review:
\usepackage{icml2022}

% If accepted, instead use the following line for the camera-ready submission:
% \usepackage[accepted]{icml2022}

% For theorems and such
\usepackage{amsmath}
\usepackage{amssymb}
\usepackage{mathtools}
\usepackage{amsthm}

% if you use cleveref..
\usepackage[capitalize,noabbrev]{cleveref}

%%%%%%%%%%%%%%%%%%%%%%%%%%%%%%%%
% THEOREMS
%%%%%%%%%%%%%%%%%%%%%%%%%%%%%%%%
\theoremstyle{plain}
\newtheorem{theorem}{Theorem}[section]
\newtheorem{proposition}[theorem]{Proposition}
\newtheorem{lemma}[theorem]{Lemma}
\newtheorem{corollary}[theorem]{Corollary}
\theoremstyle{definition}
\newtheorem{definition}[theorem]{Definition}
\newtheorem{assumption}[theorem]{Assumption}
\theoremstyle{remark}
\newtheorem{remark}[theorem]{Remark}

% Todonotes is useful during development; simply uncomment the next line
%    and comment out the line below the next line to turn off comments
%\usepackage[disable,textsize=tiny]{todonotes}
\usepackage[textsize=tiny]{todonotes}


% The \icmltitle you define below is probably too long as a header.
% Therefore, a short form for the running title is supplied here:
\icmltitlerunning{Submission and Formatting Instructions for ICML 2022}

\begin{document}

\twocolumn[
\icmltitle{Submission and Formatting Instructions for \\
           International Conference on Machine Learning (ICML 2022)}

% It is OKAY to include author information, even for blind
% submissions: the style file will automatically remove it for you
% unless you've provided the [accepted] option to the icml2022
% package.

% List of affiliations: The first argument should be a (short)
% identifier you will use later to specify author affiliations
% Academic affiliations should list Department, University, City, Region, Country
% Industry affiliations should list Company, City, Region, Country

% You can specify symbols, otherwise they are numbered in order.
% Ideally, you should not use this facility. Affiliations will be numbered
% in order of appearance and this is the preferred way.
\icmlsetsymbol{equal}{*}

\begin{icmlauthorlist}
\icmlauthor{Firstname1 Lastname1}{equal,yyy}
\icmlauthor{Firstname2 Lastname2}{equal,yyy,comp}
\icmlauthor{Firstname3 Lastname3}{comp}
\icmlauthor{Firstname4 Lastname4}{sch}
\icmlauthor{Firstname5 Lastname5}{yyy}
\icmlauthor{Firstname6 Lastname6}{sch,yyy,comp}
\icmlauthor{Firstname7 Lastname7}{comp}
%\icmlauthor{}{sch}
\icmlauthor{Firstname8 Lastname8}{sch}
\icmlauthor{Firstname8 Lastname8}{yyy,comp}
%\icmlauthor{}{sch}
%\icmlauthor{}{sch}
\end{icmlauthorlist}

\icmlaffiliation{yyy}{Department of XXX, University of YYY, Location, Country}
\icmlaffiliation{comp}{Company Name, Location, Country}
\icmlaffiliation{sch}{School of ZZZ, Institute of WWW, Location, Country}

\icmlcorrespondingauthor{Firstname1 Lastname1}{first1.last1@xxx.edu}
\icmlcorrespondingauthor{Firstname2 Lastname2}{first2.last2@www.uk}

% You may provide any keywords that you
% find helpful for describing your paper; these are used to populate
% the "keywords" metadata in the PDF but will not be shown in the document
\icmlkeywords{Machine Learning, ICML}

\vskip 0.3in
]

% this must go after the closing bracket ] following \twocolumn[ ...

% This command actually creates the footnote in the first column
% listing the affiliations and the copyright notice.
% The command takes one argument, which is text to display at the start of the footnote.
% The \icmlEqualContribution command is standard text for equal contribution.
% Remove it (just {}) if you do not need this facility.

%\printAffiliationsAndNotice{}  % leave blank if no need to mention equal contribution
\printAffiliationsAndNotice{\icmlEqualContribution} % otherwise use the standard text.

\begin{abstract}
\end{abstract}

\section{Introduction}
\section{Related works}
\section{Proposed method}
In this section the proposed method will be explained. Our new method have differences in architecture compare to Graph-Unets that are mentioned bellow:
\subsection{Pooling layer}
The main novelty of this paper is changes proposed in this layer. We propose two different pooling layers. In previous work(Graph U-nets) the main idea for this layer was finding top k important nodes by their features and that method doesn't consider the structure of graph. Our proposed layer consider both features of each nodes and structure of nodes in graph to choose top k important nodes. To move towards achieving this goal we use two approaches, first using centralities of node and second generalization spectral sampling method. Now we will describe both of these methods:
\subsubsection{Centrality based}
For choosing most important nodes of graph just from its structure, one easy way is calculating each node centrality and choose top k of them, but in different graphs and different tasks needs different centralities to be useful and efficient. So for achieving this goal and using different centralities we propose this method. In this method we consider the importance of each node with linear composition of centralities. To explain it more accurate first we augment some important and easy calculating centralities to form a matrix that we call it "centrality matrix" then this matrix multiply with learnable vector in neural network. This means the neural network depending to the task and graph learns vector that its each components show how much that centrality is important. Finally the result of multiplying centrality matrix and this vector will be the vector of importance of each node according to graph structure. we add these structure scores to feature scores comes from Graph-Unets method and choose top k nodes from these scores.
\begin{algorithm}[tb]
	\caption{Centrality Based }
	\label{alg:example}
	\begin{algorithmic}
		\STATE {\bfseries Input:} Graph $G=(V, E)$, size $m$
		\STATE Initialize $S = 0_{|V| \times m}$
		\FOR{$i=1$ {\bfseries to} $|V|$}
		\FOR{$j=1$ {\bfseries to} $m$}
		\STATE $S_{ij} =$ j-th centrality for i-th node
		\ENDFOR
		\ENDFOR
	\end{algorithmic}
\end{algorithm}
\subsubsection{General spectral sampling}
According to paper ... they proposed a method for sampling nodes with spectral decomposition. to summarize their method, they find smoothest signal and choose node with most energy for a sample of whole graph. The key part of our generalization is come from the fact that least eigenvalues are very near to each other so if we can use m-least of them maybe it leads to get a better result. At first we explain how the sampling method works, after spectral decomposition of Laplacian matrix and finding least eigenvalues and corresponding eigenvector of p-th power of the Laplacian matrix, we should replace each components of eigenvectors to absolute of it and then the index of biggest value will be the index of node we should sample from graph. To generalize the method of sampling we can calculate m-least eigenvalues and eigenvectors and after finding absolute values of each components of eigenvectors we can augment them to form a matrix that we call it "spectral matrix" and similarly to centrality based method we learn a vector. the way of how to choosing top k nodes is the same as before.
\begin{algorithm}[tb]
	\caption{Generalization of Sampling }
	\label{alg:example}
	\begin{algorithmic}
		\STATE {\bfseries Input:} Adjacency Matrix $A$, size $m$, power $p$
		\STATE $L=NormalizedLaplacian(A^p)$
		\STATE $U\Sigma V = SVD(L)$
		\STATE $idx = argsort(\lambda)$
		\STATE $idx = least(idx, m)$
		\STATE $S=V[:,idx]$
		\STATE $S=absolute(S)$
	\end{algorithmic}
\end{algorithm}
In algorithm 2 $\lambda$ is eigenvalues vector and absolute function returns matrix of absolute values of each components.
\begin{algorithm}[tb]
	\caption{Pooling }
	\label{alg:example}
	\begin{algorithmic}
		\STATE {\bfseries Input:} Features' Matrix $X^l$, Adjacency Matrix $A^l$, Structure Matrix $S^l$, Vector $p$, Vector $q$, Number $k$
		\STATE $y_{features} = X^lp/ \| p \|$
		\STATE $y_{structure} = S^lq/ \| q \|$
		\STATE $y = y_{features} + y_{structures}$
		\STATE $idx=rank(y,k)$
		\STATE $\tilde{y} = sigmoid(y[idx])$
		\STATE $\tilde{X}^{l+1}=X^l[idx, :]$
		\STATE $X^{l+1}=\tilde{X}^{l+1} \odot (\tilde{y}1^T)$
		\STATE $A^{l+1}=A^l[idx,idx]$
	\end{algorithmic}
\end{algorithm}


\subsection{Graph convolution layer}
It's well-known that GIN(Graph Isomorphism Network) is more powerful than GraphSage and GCN(Graph Convolutional Network) so it would be reasonable to replace GCN with GIN.
\subsection{Unpooling layer}
The main difference in this layer is we initialize features of each new nodes with weighted average of its neighbors' features. 
\begin{algorithm}[tb]
	\caption{Unpooling }
	\label{alg:example}
	\begin{algorithmic}
	\STATE $X^{l+1}=distribute(0_{N \times C}, X^l, idx)$
	\STATE  $I=V \textbackslash idx$
	\STATE$ \forall \: i \in I: X_i^{l+1}= \frac{\sum_{j \in V} A_{ij}X_i^{l}}{\sum_{j \in V A_{ij}}}$
	\end{algorithmic}
\end{algorithm}
\subsection{Loss function}
The loss function of these method is added with regularization term to increase accuracy in some datasets. As a result of using this regularization term is finding more smooth decision boundary. the proposed regularization term defines bellow:
\begin{gather}
	R=\frac{1}{L}\sum_{i=1}^{L}\frac{\left \|W^i\right \|_F}{\#W^icomponents}, W^i\in \mathbb{R}^{m^i\times n^i}
	\\ \Rightarrow R=\frac{1}{L}\sum_{i=1}^{L}\frac{\left \|W^i\right \|_F}{m^i\times n^i}
\end{gather}
which $W^i$ is transform matrix of i-th layer in dense network for classification.
\\In order to make clear why we choose this regularization format, we should tell that the common way for regularization would be $R=\frac{1}{L}\sum_{i=1}^{L}{\left \|W^i\right \|_F}$ but the layers dimensions are not similar to each other. to normalize these norms of each layer we divide each of them by number of their components. In other word we choose different regularization coefficient for different layers.

\section{Experiments}
\section{Conclusion}

\section*{Acknowledgements}

\textbf{Do not} include acknowledgements in the initial version of
the paper submitted for blind review.

If a paper is accepted, the final camera-ready version can (and
probably should) include acknowledgements. In this case, please
place such acknowledgements in an unnumbered section at the
end of the paper. Typically, this will include thanks to reviewers
who gave useful comments, to colleagues who contributed to the ideas,
and to funding agencies and corporate sponsors that provided financial
support.


% In the unusual situation where you want a paper to appear in the
% references without citing it in the main text, use \nocite
\nocite{langley00}

\bibliography{example_paper}
\bibliographystyle{icml2022}


%%%%%%%%%%%%%%%%%%%%%%%%%%%%%%%%%%%%%%%%%%%%%%%%%%%%%%%%%%%%%%%%%%%%%%%%%%%%%%%
%%%%%%%%%%%%%%%%%%%%%%%%%%%%%%%%%%%%%%%%%%%%%%%%%%%%%%%%%%%%%%%%%%%%%%%%%%%%%%%
% APPENDIX
%%%%%%%%%%%%%%%%%%%%%%%%%%%%%%%%%%%%%%%%%%%%%%%%%%%%%%%%%%%%%%%%%%%%%%%%%%%%%%%
%%%%%%%%%%%%%%%%%%%%%%%%%%%%%%%%%%%%%%%%%%%%%%%%%%%%%%%%%%%%%%%%%%%%%%%%%%%%%%%
\newpage
\appendix
\onecolumn
\section{You \emph{can} have an appendix here.}

You can have as much text here as you want. The main body must be at most $8$ pages long.
For the final version, one more page can be added.
If you want, you can use an appendix like this one, even using the one-column format.
%%%%%%%%%%%%%%%%%%%%%%%%%%%%%%%%%%%%%%%%%%%%%%%%%%%%%%%%%%%%%%%%%%%%%%%%%%%%%%%
%%%%%%%%%%%%%%%%%%%%%%%%%%%%%%%%%%%%%%%%%%%%%%%%%%%%%%%%%%%%%%%%%%%%%%%%%%%%%%%


\end{document}


% This document was modified from the file originally made available by
% Pat Langley and Andrea Danyluk for ICML-2K. This version was created
% by Iain Murray in 2018, and modified by Alexandre Bouchard in
% 2019 and 2021 and by Csaba Szepesvari, Gang Niu and Sivan Sabato in 2022. 
% Previous contributors include Dan Roy, Lise Getoor and Tobias
% Scheffer, which was slightly modified from the 2010 version by
% Thorsten Joachims & Johannes Fuernkranz, slightly modified from the
% 2009 version by Kiri Wagstaff and Sam Roweis's 2008 version, which is
% slightly modified from Prasad Tadepalli's 2007 version which is a
% lightly changed version of the previous year's version by Andrew
% Moore, which was in turn edited from those of Kristian Kersting and
% Codrina Lauth. Alex Smola contributed to the algorithmic style files.
